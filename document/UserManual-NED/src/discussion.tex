
% 5.  Discussion
\chapter{نتیجه گیری}

%    (1) Why use a language for describing data?  What's wrong with
%        diagrams?
% 
%    There are many advantages in using a data-description language such
%    as XDR versus using diagrams.  Languages are more formal than
%    diagrams and lead to less ambiguous descriptions of data.  Languages
%    are also easier to understand and allow one to think of other issues
%    instead of the low-level details of bit encoding.  Also, there is a
%    close analogy between the types of XDR and a high-level language such
%    as C or Pascal.  This makes the implementation of XDR encoding and
%    decoding modules an easier task.  Finally, the language specification
%    itself is an ASCII string that can be passed from machine to machine
%    to perform on-the-fly data interpretation.
% 
%    (2) Why is there only one byte-order for an XDR unit?
% 
%    Supporting two byte-orderings requires a higher-level protocol for
%    determining in which byte-order the data is encoded.  Since XDR is
%    not a protocol, this can't be done.  The advantage of this, though,
%    is that data in XDR format can be written to a magnetic tape, for
%    example, and any machine will be able to interpret it, since no
%    higher-level protocol is necessary for determining the byte-order.
% 
%    (3) Why is the XDR byte-order big-endian instead of little-endian?
%        Isn't this unfair to little-endian machines such as the VAX(r),
%        which has to convert from one form to the other?
% 
%    Yes, it is unfair, but having only one byte-order means you have to
%    be unfair to somebody.  Many architectures, such as the Motorola
%    68000* and IBM 370*, support the big-endian byte-order.
% 
%    (4) Why is the XDR unit four bytes wide?
% 
%    There is a tradeoff in choosing the XDR unit size.  Choosing a small
%    size, such as two, makes the encoded data small, but causes alignment
%    problems for machines that aren't aligned on these boundaries.  A
%    large size, such as eight, means the data will be aligned on
%    virtually every machine, but causes the encoded data to grow too big.
%    We chose four as a compromise.  Four is big enough to support most
%    architectures efficiently, except for rare machines such as the
%    eight-byte-aligned Cray*.  Four is also small enough to keep the
%    encoded data restricted to a reasonable size.
% 
%    (5) Why must variable-length data be padded with zeros?
% 
%    It is desirable that the same data encode into the same thing on all
%    machines, so that encoded data can be meaningfully compared or
%    checksummed.  Forcing the padded bytes to be zero ensures this.
% 
%    (6) Why is there no explicit data-typing?
% 
%    Data-typing has a relatively high cost for what small advantages it
%    may have.  One cost is the expansion of data due to the inserted type
%    fields.  Another is the added cost of interpreting these type fields
%    and acting accordingly.  And most protocols already know what type
%    they expect, so data-typing supplies only redundant information.
%    However, one can still get the benefits of data-typing using XDR.
%    One way is to encode two things: first, a string that is the XDR data
%    description of the encoded data, and then the encoded data itself.
%    Another way is to assign a value to all the types in XDR, and then
%    define a universal type that takes this value as its discriminant and
%    for each value, describes the corresponding data type.
