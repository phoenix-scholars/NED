
\chapter{آرایه}

% Fixed-Length Array
\section{آرایه با طول ثابت}

%    Declarations for fixed-length arrays of homogeneous elements are in
%    the following form:
% 
%          type-name identifier[n];
% 
%    Fixed-length arrays of elements numbered 0 through n-1 are encoded by
%    individually encoding the elements of the array in their natural
%    order, 0 through n-1.  Each element's size is a multiple of four
%    bytes.  Though all elements are of the same type, the elements may
%    have different sizes.  For example, in a fixed-length array of
%    strings, all elements are of type "string", yet each element will
%    vary in its length.
% 
%          +---+---+---+---+---+---+---+---+...+---+---+---+---+
%          |   element 0   |   element 1   |...|  element n-1  |
%          +---+---+---+---+---+---+---+---+...+---+---+---+---+
%          |<--------------------n elements------------------->|
% 
%                                                FIXED-LENGTH ARRAY

% Variable-Length Array
\section{آرایه با طول متغیر}

%    Counted arrays provide the ability to encode variable-length arrays
%    of homogeneous elements.  The array is encoded as the element count n
%    (an unsigned integer) followed by the encoding of each of the array's
%    elements, starting with element 0 and progressing through element
%    n-1.  The declaration for variable-length arrays follows this form:
آرایه با طول متغیر امکان کدگزاری داده با طول متغیر را فراهم می‌کند.
طول یک آرایه با استفاده از عدد صحیح بدون علامت نمایش داده می شود.
طول آرایه همواره با یک عدد صحیح بدون علامت نمایش داده می‌شود از این رو تنها
انواع داده بدون علامت و صحیح می‌تواند برای تعیین طول آرایه به کار رود.

\begin{C++}
type-name identifier<length-declaration>;
\end{C++}

انواع داده زیر می‌تواند به عنوان متغیر طول آرایه به کار گرفته شود:

\begin{itemize}
  \item \lr{unsigned byte}
  \item \lr{unsigned short int}
  \item \lr{unsigned int}
  \item \lr{unsigned long int}
\end{itemize}

برای نمونه یک آرایه از بایت‌ها که حداکثر ۲۵۵ بایت را در خود جای می‌دهد به صورت
زیر تعریف می‌شود:

\begin{C++}
byte data<unsigned byte length>;
\end{C++}

در کدگزاری یک آرایه با طول متغیر ابتدا طول آرایه کدگزاری می‌شود و در جریان بایتی
نوشته می‌شود، پس از آن هر عنصر آرایه به ترتیب اندیس آن کدگزاری شده و در جریان
بایتی نوشته می شود.
شکل \ref{image/array/variable-length} روش کدگزاری آرایه با طول متغیر را نمایش
داده است.

%            0  1  2  3
%          +--+--+--+--+--+--+--+--+--+--+--+--+...+--+--+--+--+
%          |     n     | element 0 | element 1 |...|element n-1|
%          +--+--+--+--+--+--+--+--+--+--+--+--+...+--+--+--+--+
%          |<-4 bytes->|<--------------n elements------------->|

\begin{figure}
\centering
\includegraphics[width=0.8\textwidth]{image/array/variable-length}
\caption[کدگذاری یک آرایه با طول متغیر]{
	کدگزاری یک آرایه با طول متغیر. متغیر طول آرایه در ابتدای جریان بایتی کدگزاری
	می‌شود و بعد از آن عنصرها به ترتیب اندیس کدگزاری خواهند شد.
}
\label{image/array/variable-length}
\end{figure}

\begin{warning}
استفاده از مقادیر غیر مجاز (بیش از گنجایش متغیر طول) و یا متغیرهای صحیح علامت
دار به عنوان طول خطا به شمار می‌آید.
\end{warning}


% 4.11.  String
\section{رشته}

%    The standard defines a string of n (numbered 0 through n-1) ASCII
%    bytes to be the number n encoded as an unsigned integer (as described
%    above), and followed by the n bytes of the string.  Byte m of the
%    string always precedes byte m+1 of the string, and byte 0 of the
%    string always follows the string's length.  If n is not a multiple of
%    four, then the n bytes are followed by enough (0 to 3) residual zero
%    bytes, r, to make the total byte count a multiple of four.  Counted
%    byte strings are declared as follows:
% 
%          string object<m>;
%       or
%          string object<>;
% 
%    The constant m denotes an upper bound of the number of bytes that a
%    string may contain.  If m is not specified, as in the second
%    declaration, it is assumed to be (2**32) - 1, the maximum length.
%    The constant m would normally be found in a protocol specification.
%    For example, a filing protocol may state that a file name can be no
%    longer than 255 bytes, as follows:
% 
%          string filename<255>;
% 
%             0     1     2     3     4     5   ...
%          +-----+-----+-----+-----+-----+-----+...+-----+-----+...+-----+
%          |        length n       |byte0|byte1|...| n-1 |  0  |...|  0  |
%          +-----+-----+-----+-----+-----+-----+...+-----+-----+...+-----+
%          |<-------4 bytes------->|<------n bytes------>|<---r bytes--->|
%                                  |<----n+r (where (n+r) mod 4 = 0)---->|
%                                                                   STRING
% 
%    It is an error to encode a length greater than the maximum described
%    in the specification.

