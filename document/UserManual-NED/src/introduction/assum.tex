\section{فرض‌ها}

در استاندارد ند دسته‌ای از اصول به عنوان فرض اولیه در نظر گرفته شده است. در این
بخش فرض‌های این استاندار تشریح شده است.

کوچکترین واحد داده بایت (۸ بیت) در نظر گرفته می‌شود که نمایش آن در تمام
سخت‌افزارها و سیستم‌های نرم‌افزاری یکسان است.
از این رو تمام سخت افزارها باید یک بایت داده را به گونه‌ای کدگذاری کنند که در
تمام سخت افزارها به یک شکل کد گذاری می‌شود و نتوان از آن معنی دیگری برداشت کرد.
برای نمونه در پردازنده‌ها یک بایت به صورت \glspl{big-endian} کدگذاری می‌شود، از
این رو تمام سخت افزارهای دیگری که از این استاندارد استفاده می‌کنند باید این نوع
کدگذاری را استفاده کنند.

\begin{note}
این فرض به این معنی است که تمام نهادهایی که از این استاندارد استفاده می‌کنند
باید به صورت یکتا کدگزاری یک بایت داده را انجام دهند.
\end{note}
