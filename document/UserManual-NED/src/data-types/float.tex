
% 4.6.  Floating-Point
\section{ممیز شناور}

%    The standard defines the floating-point data type "float" (32 bits or
%    4 bytes).  The encoding used is the IEEE standard for normalized
%    single-precision floating-point numbers [IEEE].  The following three
%    fields describe the single-precision floating-point number:
% 
%       S: The sign of the number.  Values 0 and 1 represent positive and
%          negative, respectively.  One bit.
% 
%       E: The exponent of the number, base 2.  8 bits are devoted to this
%          field.  The exponent is biased by 127.
% 
%       F: The fractional part of the number's mantissa, base 2.  23 bits
%          are devoted to this field.
% 
%    Therefore, the floating-point number is described by:
% 
%          (-1)**S * 2**(E-Bias) * 1.F
% 
%    It is declared as follows:
% 
%          float identifier;
% 
%          +-------+-------+-------+-------+
%          |byte 0 |byte 1 |byte 2 |byte 3 |              SINGLE-PRECISION
%          S|   E   |           F          |         FLOATING-POINT NUMBER
%          +-------+-------+-------+-------+
%          1|<- 8 ->|<-------23 bits------>|
%          <------------32 bits------------>
% 
%    Just as the most and least significant bytes of a number are 0 and 3,
%    the most and least significant bits of a single-precision floating-
%    point number are 0 and 31.  The beginning bit (and most significant
%    bit) offsets of S, E, and F are 0, 1, and 9, respectively.  Note that
%    these numbers refer to the mathematical positions of the bits, and
%    NOT to their actual physical locations (which vary from medium to
%    medium).
% 
%    The IEEE specifications should be consulted concerning the encoding
%    for signed zero, signed infinity (overflow), and denormalized numbers
%    (underflow) [IEEE].  According to IEEE specifications, the "NaN" (not
%    a number) is system dependent and should not be interpreted within
%    XDR as anything other than "NaN".
