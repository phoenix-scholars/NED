


% 4.1.  Integer
\section{عدد صحیح}

%    An XDR signed integer is a 32-bit datum that encodes an integer in
%    the range [-2147483648,2147483647].  The integer is represented in
%    two's complement notation.  The most and least significant bytes are
%    0 and 3, respectively.  Integers are declared as follows:

در استاندارد ند، عدد صیحیح علامت‌دار با استفاده از یک داده ۳۲ بیتی کد گذاری
می‌شود که قادر است عددهای صحیح در بازه $[-2147483648,2147483647]$ را در خود جای
دهد.
عدد صحیح با استفاده از متمم دو نمایش داده می‌شود.
پر ارزش‌ترین بایت بایت شماره ۳ و کم ‌ارزش‌ترین بایت بایت شماره صفر است.
در شکل \ref{image/data-types/integer} ساختار کلی جریان داده برای کدگزاری یک عدد
صحیح نمایش داده شده است.

%          int identifier;
% 
%            (MSB)                   (LSB)
%          +-------+-------+-------+-------+
%          |byte 0 |byte 1 |byte 2 |byte 3 |                      INTEGER
%          +-------+-------+-------+-------+
%          <------------32 bits------------>

\begin{figure}
\centering
\includegraphics[width=0.8\textwidth]{image/data-types/integer}
\caption[کدگذاری یک داده صحیح]{
	ساختار داده‌ای یک داده صحیح شامل ۴ بایت است که برای نمایش یک عدد صحیح ۳۲ بیتی
	به کار گرفته می‌شود. در این ساختار بایت صفر کم ارزش‌ترین و بایت ۳ پر ارزش‌ترین
	بایت است.
}
\label{image/data-types/integer}
\end{figure}


یک عدد صیحی به صورت زیر تعریف می‌شود:

\begin{C++}
int identifier;
\end{C++}
