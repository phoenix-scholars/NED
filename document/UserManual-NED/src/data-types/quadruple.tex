
% 4.8.  Quadruple-Precision Floating-Point
\section{ممیز شناور چهارلا}
% 
%    The standard defines the encoding for the quadruple-precision
%    floating-point data type "quadruple" (128 bits or 16 bytes).  The
%    encoding used is designed to be a simple analog of the encoding used
%    for single- and double-precision floating-point numbers using one
%    form of IEEE double extended precision.  The standard encodes the
%    following three fields, which describe the quadruple-precision
%    floating-point number:
% 
%       S: The sign of the number.  Values 0 and 1 represent positive and
%          negative, respectively.  One bit.
% 
%       E: The exponent of the number, base 2.  15 bits are devoted to
%          this field.  The exponent is biased by 16383.
% 
%       F: The fractional part of the number's mantissa, base 2.  112 bits
%          are devoted to this field.
% 
%    Therefore, the floating-point number is described by:
% 
%          (-1)**S * 2**(E-Bias) * 1.F
% 
%    It is declared as follows:
% 
%          quadruple identifier;
% 
%          +------+------+------+------+------+------+-...--+------+
%          |byte 0|byte 1|byte 2|byte 3|byte 4|byte 5| ...  |byte15|
%          S|    E       |                  F                      |
%          +------+------+------+------+------+------+-...--+------+
%          1|<----15---->|<-------------112 bits------------------>|
%          <-----------------------128 bits------------------------>
%                                       QUADRUPLE-PRECISION FLOATING-POINT
% 
%    Just as the most and least significant bytes of a number are 0 and 3,
%    the most and least significant bits of a quadruple-precision
%    floating-point number are 0 and 127.  The beginning bit (and most
%    significant bit) offsets of S, E , and F are 0, 1, and 16,
%    respectively.  Note that these numbers refer to the mathematical
%    positions of the bits, and NOT to their actual physical locations
%    (which vary from medium to medium).

%    The encoding for signed zero, signed infinity (overflow), and
%    denormalized numbers are analogs of the corresponding encodings for
%    single and double-precision floating-point numbers [SPAR], [HPRE].
%    The "NaN" encoding as it applies to quadruple-precision floating-
%    point numbers is system dependent and should not be interpreted
%    within XDR as anything other than "NaN".
