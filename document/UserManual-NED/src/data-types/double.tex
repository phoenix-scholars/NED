
% 4.7.  Double-Precision Floating-Point
\section{ممیز شناور دوبل}

%    The standard defines the encoding for the double-precision floating-
%    point data type "double" (64 bits or 8 bytes).  The encoding used is
%    the IEEE standard for normalized double-precision floating-point
%    numbers [IEEE].  The standard encodes the following three fields,
%    which describe the double-precision floating-point number:
% 
%       S: The sign of the number.  Values 0 and 1 represent positive and
%             negative, respectively.  One bit.
% 
%       E: The exponent of the number, base 2.  11 bits are devoted to
%             this field.  The exponent is biased by 1023.
% 
%       F: The fractional part of the number's mantissa, base 2.  52 bits
%             are devoted to this field.
% 
%    Therefore, the floating-point number is described by:
% 
%          (-1)**S * 2**(E-Bias) * 1.F
% 
%    It is declared as follows:
% 
%          double identifier;
% 
%          +------+------+------+------+------+------+------+------+
%          |byte 0|byte 1|byte 2|byte 3|byte 4|byte 5|byte 6|byte 7|
%          S|    E   |                    F                        |
%          +------+------+------+------+------+------+------+------+
%          1|<--11-->|<-----------------52 bits------------------->|
%          <-----------------------64 bits------------------------->
%                                         DOUBLE-PRECISION FLOATING-POINT
% 
%    Just as the most and least significant bytes of a number are 0 and 3,
%    the most and least significant bits of a double-precision floating-
%    point number are 0 and 63.  The beginning bit (and most significant
%    bit) offsets of S, E, and F are 0, 1, and 12, respectively.  Note
%    that these numbers refer to the mathematical positions of the bits,
%    and NOT to their actual physical locations (which vary from medium to
%    medium).

%    The IEEE specifications should be consulted concerning the encoding
%    for signed zero, signed infinity (overflow), and denormalized numbers
%    (underflow) [IEEE].  According to IEEE specifications, the "NaN" (not
%    a number) is system dependent and should not be interpreted within
%    XDR as anything other than "NaN".