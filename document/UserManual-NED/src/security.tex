
% 8.  Security Considerations
\chapter{امنیت و ملاحظات آن}
% 
%    XDR is a data description language, not a protocol, and hence it does
%    not inherently give rise to any particular security considerations.
%    Protocols that carry XDR-formatted data, such as NFSv4, are
%    responsible for providing any necessary security services to secure
%    the data they transport.
% 
%    Care must be take to properly encode and decode data to avoid
%    attacks.  Known and avoidable risks include:
% 
%    *    Buffer overflow attacks.  Where feasible, protocols should be
%         defined with explicit limits (via the "<" [ value ] ">" notation
%         instead of "<" ">") on elements with variable-length data types.
%         Regardless of the feasibility of an explicit limit on the
%         variable length of an element of a given protocol, decoders need
%         to ensure the incoming size does not exceed the length of any
%         provisioned receiver buffers.
% 
%    *    Nul octets embedded in an encoded value of type string.  If the
%         decoder's native string format uses nul-terminated strings, then
%         the apparent size of the decoded object will be less than the
%         amount of memory allocated for the string.  Some memory
%         deallocation interfaces take a size argument.  The caller of the
%         deallocation interface would likely determine the size of the
%         string by counting to the location of the nul octet and adding
%         one.  This discrepancy can cause memory leakage (because less
%         memory is actually returned to the free pool than allocated),
%         leading to system failure and a denial of service attack.
% 
%    *    Decoding of characters in strings that are legal ASCII
%         characters but nonetheless are illegal for the intended
%         application.  For example, some operating systems treat the '/'
%         character as a component separator in path names.  For a
%         protocol that encodes a string in the argument to a file
%         creation operation, the decoder needs to ensure that '/' is not
%         inside the component name.  Otherwise, a file with an illegal
%         '/' in its name will be created, making it difficult to remove,
%         and is therefore a denial of service attack.
% 
%    *    Denial of service caused by recursive decoder or encoder
%         subroutines.  A recursive decoder or encoder might process data
%         that has a structured type with a member of type optional data
%         that directly or indirectly refers to the structured type (i.e.,
%         a linked list).  For example,
% 
%               struct m {
%                 int x;
%                 struct m *next;
%               };
% 
%         An encoder or decoder subroutine might be written to recursively
%         call itself each time another element of type "struct m" is
%         found.  An attacker could construct a long linked list of
%         "struct m" elements in the request or response, which then
%         causes a stack overflow on the decoder or encoder.  Decoders and
%         encoders should be written non-recursively or impose a limit on
%         list length.