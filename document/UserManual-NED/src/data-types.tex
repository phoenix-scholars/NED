
% 4.  Data Types
\chapter{نوع داده‌}

%    Each of the sections that follow describes a data type defined in the
%    XDR standard, shows how it is declared in the language, and includes
%    a graphic illustration of its encoding.

در این بخش انواع داده‌های از پیش تعریف شده در استاندارد ند تشریح شده است.
هر ساختار داده با استفاده از شکل گرافیکی تشریح شده است که بیانگر نحوه ذخیره سازی
آن در جریان داده است.
 
%    For each data type in the language we show a general paradigm
%    declaration.  Note that angle brackets (< and >) denote variable-
%    length sequences of data and that square brackets ([ and ]) denote
%    fixed-length sequences of data.  "n", "m", and "r" denote integers.
%    For the full language specification and more formal definitions of
%    terms such as "identifier" and "declaration", refer to Section 6,
%    "The XDR Language Specification".
%
%    For some data types, more specific examples are included.  A more
%    extensive example of a data description is in Section 7, "An Example
%    of an XDR Data Description".



% 4.16.  Void
\section{تهی}

%    An XDR void is a 0-byte quantity.  Voids are useful for describing
%    operations that take no data as input or no data as output.  They are
%    also useful in unions, where some arms may contain data and others do
%    not.  The declaration is simply as follows:

تهی معادل با داده‌ای به طول صفر است. داده تهی برای توصیف عمل‌هایی که هیچ داده
خروجی و یا ورودی ندارند پر کاربرد است.
در برخی از موارد که قسمت‌هایی از ساختارهای داده‌ای باید خالی باشد، داده تهی
کاربرد دارد.
تعریف یک داده تهی به صورت زیر است:

\begin{C++}
void;
\end{C++}

% 
%    Voids are illustrated as follows:
% 
%            ++
%            ||                                                     VOID
%            ++
%          --><-- 0 bytes

\section{بایت}

کوچکترین واحد داده در استاندارد ند بایت است. در استاندارد هیچ کدگذاری برای یک
بایت در نظر گرفته نمی‌شود.
در حالت کلی یک بایت به صورت زیر تعریف می‌شود:

\begin{C++}
byte identifier;
\end{C++}

\section{بایت بدون علامت}

کوچکترین واحد داده در استاندارد ند بایت است که معادل با یک عدد صحیح بدون علامت
در نظر گرفته می‌شود.
در حالت کلی یک بایت به صورت زیر تعریف می‌شود:

\begin{C++}
unsigned byte identifier;
\end{C++}




% 4.1.  Integer
\section{عدد صحیح}

%    An XDR signed integer is a 32-bit datum that encodes an integer in
%    the range [-2147483648,2147483647].  The integer is represented in
%    two's complement notation.  The most and least significant bytes are
%    0 and 3, respectively.  Integers are declared as follows:

در استاندارد ند، عدد صیحیح علامت‌دار با استفاده از یک داده ۳۲ بیتی کد گذاری
می‌شود که قادر است عددهای صحیح در بازه $[-2147483648,2147483647]$ را در خود جای
دهد.
عدد صحیح با استفاده از متمم دو نمایش داده می‌شود.
پر ارزش‌ترین بایت بایت شماره ۳ و کم ‌ارزش‌ترین بایت بایت شماره صفر است.
در شکل \ref{image/data-types/integer} ساختار کلی جریان داده برای کدگزاری یک عدد
صحیح نمایش داده شده است.

%          int identifier;
% 
%            (MSB)                   (LSB)
%          +-------+-------+-------+-------+
%          |byte 0 |byte 1 |byte 2 |byte 3 |                      INTEGER
%          +-------+-------+-------+-------+
%          <------------32 bits------------>

\begin{figure}
\centering
\includegraphics[width=0.8\textwidth]{image/data-types/integer}
\caption[کدگذاری یک داده صحیح]{
	ساختار داده‌ای یک داده صحیح شامل ۴ بایت است که برای نمایش یک عدد صحیح ۳۲ بیتی
	به کار گرفته می‌شود. در این ساختار بایت صفر کم ارزش‌ترین و بایت ۳ پر ارزش‌ترین
	بایت است.
}
\label{image/data-types/integer}
\end{figure}


یک عدد صیحی به صورت زیر تعریف می‌شود:

\begin{C++}
int identifier;
\end{C++}

% 4.2.  Unsigned Integer
\section{عدد صحیح بدون علامت}

%    An XDR unsigned integer is a 32-bit datum that encodes a non-negative
%    integer in the range [0,4294967295].  It is represented by an
%    unsigned binary number whose most and least significant bytes are 0
%    and 3, respectively.  An unsigned integer is declared as follows:
% 
%          unsigned int identifier;
% 
%            (MSB)                   (LSB)
%             +-------+-------+-------+-------+
%             |byte 0 |byte 1 |byte 2 |byte 3 |           UNSIGNED INTEGER
%             +-------+-------+-------+-------+
%             <------------32 bits------------>


\begin{figure}
\centering
\includegraphics[width=0.8\textwidth]{image/data-types/unsigned-integer}
\caption[ساختار یک داده صحیح بدون علامت]{
	ساختار داده‌ای یک داده صحیح بدون علامت شامل ۴ بایت است. در این ساختار بایت صفر
	کم ارزش‌ترین و بایت ۳ پر ارزش‌ترین بایت است.
}
\label{image/data-types/unsigned-integer}
\end{figure}







% 4.3.  Enumeration
\section{شماره}

%    Enumerations have the same representation as signed integers.
%    Enumerations are handy for describing subsets of the integers.
%    Enumerated data is declared as follows:
% 
%          enum { name-identifier = constant, ... } identifier;
% 
%    For example, the three colors red, yellow, and blue could be
%    described by an enumerated type:
% 
%          enum { RED = 2, YELLOW = 3, BLUE = 5 } colors;
% 
%    It is an error to encode as an enum any integer other than those that
%    have been given assignments in the enum declaration.


% 4.4.  Boolean
\section{دودویی}

%    Booleans are important enough and occur frequently enough to warrant
%    their own explicit type in the standard.  Booleans are declared as
%    follows:
% 
%          bool identifier;
% 
%    This is equivalent to:
% 
%          enum { FALSE = 0, TRUE = 1 } identifier;
\input{data-types/long-integer}


% 4.5.  Hyper Integer and Unsigned Hyper Integer
\section{صحیح بزرگ بدون علامت}

%    The standard also defines 64-bit (8-byte) numbers called hyper
%    integers and unsigned hyper integers.  Their representations are the
%    obvious extensions of integer and unsigned integer defined above.
%    They are represented in two's complement notation.  The most and
%    least significant bytes are 0 and 7, respectively.  Their
%    declarations:
% 
%    hyper identifier; unsigned hyper identifier;
% 
% 
%         (MSB)                                                   (LSB)
%       +-------+-------+-------+-------+-------+-------+-------+-------+
%       |byte 0 |byte 1 |byte 2 |byte 3 |byte 4 |byte 5 |byte 6 |byte 7 |
%       +-------+-------+-------+-------+-------+-------+-------+-------+
%       <----------------------------64 bits---------------------------->
%                                                  HYPER INTEGER
%                                                  UNSIGNED HYPER INTEGER


% 4.6.  Floating-Point
\section{ممیز شناور}

%    The standard defines the floating-point data type "float" (32 bits or
%    4 bytes).  The encoding used is the IEEE standard for normalized
%    single-precision floating-point numbers [IEEE].  The following three
%    fields describe the single-precision floating-point number:
% 
%       S: The sign of the number.  Values 0 and 1 represent positive and
%          negative, respectively.  One bit.
% 
%       E: The exponent of the number, base 2.  8 bits are devoted to this
%          field.  The exponent is biased by 127.
% 
%       F: The fractional part of the number's mantissa, base 2.  23 bits
%          are devoted to this field.
% 
%    Therefore, the floating-point number is described by:
% 
%          (-1)**S * 2**(E-Bias) * 1.F
% 
%    It is declared as follows:
% 
%          float identifier;
% 
%          +-------+-------+-------+-------+
%          |byte 0 |byte 1 |byte 2 |byte 3 |              SINGLE-PRECISION
%          S|   E   |           F          |         FLOATING-POINT NUMBER
%          +-------+-------+-------+-------+
%          1|<- 8 ->|<-------23 bits------>|
%          <------------32 bits------------>
% 
%    Just as the most and least significant bytes of a number are 0 and 3,
%    the most and least significant bits of a single-precision floating-
%    point number are 0 and 31.  The beginning bit (and most significant
%    bit) offsets of S, E, and F are 0, 1, and 9, respectively.  Note that
%    these numbers refer to the mathematical positions of the bits, and
%    NOT to their actual physical locations (which vary from medium to
%    medium).
% 
%    The IEEE specifications should be consulted concerning the encoding
%    for signed zero, signed infinity (overflow), and denormalized numbers
%    (underflow) [IEEE].  According to IEEE specifications, the "NaN" (not
%    a number) is system dependent and should not be interpreted within
%    XDR as anything other than "NaN".


% 4.7.  Double-Precision Floating-Point
\section{ممیز شناور دوبل}

%    The standard defines the encoding for the double-precision floating-
%    point data type "double" (64 bits or 8 bytes).  The encoding used is
%    the IEEE standard for normalized double-precision floating-point
%    numbers [IEEE].  The standard encodes the following three fields,
%    which describe the double-precision floating-point number:
% 
%       S: The sign of the number.  Values 0 and 1 represent positive and
%             negative, respectively.  One bit.
% 
%       E: The exponent of the number, base 2.  11 bits are devoted to
%             this field.  The exponent is biased by 1023.
% 
%       F: The fractional part of the number's mantissa, base 2.  52 bits
%             are devoted to this field.
% 
%    Therefore, the floating-point number is described by:
% 
%          (-1)**S * 2**(E-Bias) * 1.F
% 
%    It is declared as follows:
% 
%          double identifier;
% 
%          +------+------+------+------+------+------+------+------+
%          |byte 0|byte 1|byte 2|byte 3|byte 4|byte 5|byte 6|byte 7|
%          S|    E   |                    F                        |
%          +------+------+------+------+------+------+------+------+
%          1|<--11-->|<-----------------52 bits------------------->|
%          <-----------------------64 bits------------------------->
%                                         DOUBLE-PRECISION FLOATING-POINT
% 
%    Just as the most and least significant bytes of a number are 0 and 3,
%    the most and least significant bits of a double-precision floating-
%    point number are 0 and 63.  The beginning bit (and most significant
%    bit) offsets of S, E, and F are 0, 1, and 12, respectively.  Note
%    that these numbers refer to the mathematical positions of the bits,
%    and NOT to their actual physical locations (which vary from medium to
%    medium).

%    The IEEE specifications should be consulted concerning the encoding
%    for signed zero, signed infinity (overflow), and denormalized numbers
%    (underflow) [IEEE].  According to IEEE specifications, the "NaN" (not
%    a number) is system dependent and should not be interpreted within
%    XDR as anything other than "NaN".

% 4.8.  Quadruple-Precision Floating-Point
\section{ممیز شناور چهارلا}
% 
%    The standard defines the encoding for the quadruple-precision
%    floating-point data type "quadruple" (128 bits or 16 bytes).  The
%    encoding used is designed to be a simple analog of the encoding used
%    for single- and double-precision floating-point numbers using one
%    form of IEEE double extended precision.  The standard encodes the
%    following three fields, which describe the quadruple-precision
%    floating-point number:
% 
%       S: The sign of the number.  Values 0 and 1 represent positive and
%          negative, respectively.  One bit.
% 
%       E: The exponent of the number, base 2.  15 bits are devoted to
%          this field.  The exponent is biased by 16383.
% 
%       F: The fractional part of the number's mantissa, base 2.  112 bits
%          are devoted to this field.
% 
%    Therefore, the floating-point number is described by:
% 
%          (-1)**S * 2**(E-Bias) * 1.F
% 
%    It is declared as follows:
% 
%          quadruple identifier;
% 
%          +------+------+------+------+------+------+-...--+------+
%          |byte 0|byte 1|byte 2|byte 3|byte 4|byte 5| ...  |byte15|
%          S|    E       |                  F                      |
%          +------+------+------+------+------+------+-...--+------+
%          1|<----15---->|<-------------112 bits------------------>|
%          <-----------------------128 bits------------------------>
%                                       QUADRUPLE-PRECISION FLOATING-POINT
% 
%    Just as the most and least significant bytes of a number are 0 and 3,
%    the most and least significant bits of a quadruple-precision
%    floating-point number are 0 and 127.  The beginning bit (and most
%    significant bit) offsets of S, E , and F are 0, 1, and 16,
%    respectively.  Note that these numbers refer to the mathematical
%    positions of the bits, and NOT to their actual physical locations
%    (which vary from medium to medium).

%    The encoding for signed zero, signed infinity (overflow), and
%    denormalized numbers are analogs of the corresponding encodings for
%    single and double-precision floating-point numbers [SPAR], [HPRE].
%    The "NaN" encoding as it applies to quadruple-precision floating-
%    point numbers is system dependent and should not be interpreted
%    within XDR as anything other than "NaN".



% 4.17.  Constant
\section{ثابت}

% 
%    The data declaration for a constant follows this form:

تعریف یک داده ثابت به صورت زیر است:

\begin{C++}
const name-identifier = n;
\end{C++}

%    "const" is used to define a symbolic name for a constant; it does not
%    declare any data.  The symbolic constant may be used anywhere a
%    regular constant may be used.  For example, the following defines a
%    symbolic constant DOZEN, equal to 12.
ثابت برای تعریف یک مقدار ثابت به صورت یک سمبل است و هیچ نوع داده جدیدی را ایجاد
نمی‌کند.
یک ثاب را می‌توان هرجایی که یک مقدار ثابت کاربرد دارد استفاده کرد.
برای نمونه عبارت زیر یک مقدار ثابت برابر با عدد ۱۲ را تعریف می‌کند:

\begin{C++}
const DOZEN = 12;
\end{C++}



% 4.20.  Areas for Future Enhancement
% 
%    The XDR standard lacks representations for bit fields and bitmaps,
%    since the standard is based on bytes.  Also missing are packed (or
%    binary-coded) decimals.
% 
%    The intent of the XDR standard was not to describe every kind of data
%    that people have ever sent or will ever want to send from machine to
%    machine.  Rather, it only describes the most commonly used data-types
%    of high-level languages such as Pascal or C so that applications
%    written in these languages will be able to communicate easily over
%    some medium.
% 
%    One could imagine extensions to XDR that would let it describe almost
%    any existing protocol, such as TCP.  The minimum necessary for this
%    is support for different block sizes and byte-orders.  The XDR
%    discussed here could then be considered the 4-byte big-endian member
%    of a larger XDR family.
