
\chapter{دیباچه}
% 1.  Introduction

%    XDR is a standard for the description and encoding of data.  It is
%    useful for transferring data between different computer
%    architectures, and it has been used to communicate data between such
%    diverse machines as the SUN WORKSTATION*, VAX*, IBM-PC*, and Cray*.
%    XDR fits into the ISO presentation layer and is roughly analogous in
%    purpose to X.409, ISO Abstract Syntax Notation.  The major difference
%    between these two is that XDR uses implicit typing, while X.409 uses
%    explicit typing.

ند یک استاندارد برای توصیف روش‌های کدگذاری است که در سیستم‌های پردازشی مورد
استفاده قرار می‌گیرد.
ند برای انتقال داده‌ها در شبکه‌های رایانه‌ای و یا سخت‌افزارها، که در آنها معماری
نهادهای موجود در ارتباط با یکدیگر متفاوت است بسیار کاربرد دارد.
برای نمونه حالتی تصور کنید که در آن رایانه‌ای شخصی با استفاده از گذرگاه‌های
\lr{PCI} با یک ریزپردازنده در ارتباط است. 
بدیهی است که معماری این درو پردازنده با یکدیگر متفاوت است و استفاده از ساختارهای
داده عمومی نمی‌تواند در ارتباط بین آنها کاربرد داشته باشد.
ند استانداردی است که کدگذاری این نوع ارتباط را تعریف می‌کند.
استاندارد ند مشابه با استانداردهای \lr{XDR}\cite{srinivasan1995xdr} و \lr{X.409}\cite{x409}
است.
\lr{X.409} که در لایه نمایش پشته پرتوکلی \lr{ISO} کاربرد دارد، انواع داده را به
صورت صریح بیان می‌کند.
این درحالی است که استاندارد \lr{XDR} انواع داده را به صورت مجازی تعریف می‌کند و
در استانداردهای متفاوت مانند \lr{IEE1609} به کار گرفته شده است.
استاندارد ند نیز مانند استاندارد \lr{XDR} داده‌ها را به صورت مجازی نمایش می‌دهد
با این تفاوت که مدل نمایش فیزیکی داده در این استاندارد به ساختارهای فیزیکی و
پردازنده‌ها نزدیکتر است.

%    XDR uses a language to describe data formats.  The language can be
%    used only to describe data; it is not a programming language.  This
%    language allows one to describe intricate data formats in a concise
%    manner.  The alternative of using graphical representations (itself
%    an informal language) quickly becomes incomprehensible when faced
%    with complexity.  The XDR language itself is similar to the C
%    language [KERN], just as Courier [COUR] is similar to Mesa.
%    Protocols such as ONC RPC (Remote Procedure Call) and the NFS*
%    (Network File System) use XDR to describe the format of their data.

ند در حقیقت زبانی است که برای توصیف ساختارهای داده‌ای به کار گرفته می‌شود. این
زبان تنها برای توصیف داده به کار گرفته می‌شود و نمی‌توان آن را به عنوان یک زبان
برنامه سازی در نظر گرفت.
استفاده از زبان‌های گرافیکی برای نمایش داده‌ها، در قراردادها و ارتباط‌های فیزیکی
نارسایی‌های زیادی دارد.
برای نمونه در این نوع زبان‌ها نحوه قرار گرفتن داده‌ها به صورت یک جریان بیتی مشخص
نیست.
زبان ند بسیار شبیه به زبان برنامه سازی \lr{C} است با این تفاوت که ساختارهای
داده‌ای خاصی را توسعه داده و اصول یکتایی برای تبدیل این ساختارهای به رشته بیت را
مشخص کرده است.
این زبان در کاربردهایی مانند، قراردادهای شبکه، فراخوانی دور روال‌ها، انتقال داده
بین دستگاه‌های سخت افزاری و محاسبات توزیعی کابرد فراوان دارد.

\section{فرض‌ها}

در استاندارد ند دسته‌ای از اصول به عنوان فرض اولیه در نظر گرفته شده است. در این
بخش فرض‌های این استاندار تشریح شده است.

کوچکترین واحد داده بایت (۸ بیت) در نظر گرفته می‌شود که نمایش آن در تمام
سخت‌افزارها و سیستم‌های نرم‌افزاری یکسان است.
از این رو تمام سخت افزارها باید یک بایت داده را به گونه‌ای کدگذاری کنند که در
تمام سخت افزارها به یک شکل کد گذاری می‌شود و نتوان از آن معنی دیگری برداشت کرد.
برای نمونه در پردازنده‌ها یک بایت به صورت \glspl{big-endian} کدگذاری می‌شود، از
این رو تمام سخت افزارهای دیگری که از این استاندارد استفاده می‌کنند باید این نوع
کدگذاری را استفاده کنند.

\begin{note}
این فرض به این معنی است که تمام نهادهایی که از این استاندارد استفاده می‌کنند
باید به صورت یکتا کدگزاری یک بایت داده را انجام دهند.
\end{note}


% 2.  Changes from RFC 1832
\section{تاریخچه}

%    This document makes no technical changes to RFC 1832 and is published
%    for the purposes of noting IANA considerations, augmenting security
%    considerations, and distinguishing normative from informative
%    references.

استاندارد ند یک توسعه از استاندارد \lr{XDR} است، که در آن اصول نمایش داده به
شکلی تغییر کرده که با ساختارهای پردازنده‌ها و ریزپردازنده‌ها منطبق باشد.

\begin{table}
\begin{centering}
\begin{tabular}{|c|r|p{5cm}|}\hline
	نسخه &
	تاریخ &
	تغییر\\\hline
	1.0 &
	آذر 1392 &
	ارائه اولیه قرارداد نمایش داده (ند)\\\hline
\end{tabular}
\label{introduction/history}
\caption[تاریخچه تغییرات سند]{
	تاریخچه تغییرات سند.
}
\end{centering}
\end{table}



