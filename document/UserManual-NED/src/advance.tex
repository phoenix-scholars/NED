
\chapter{ساختارهای داده پیشرفته}


\section{تعرف نوع داده}

%    "typedef" does not declare any data either, but serves to define new
%    identifiers for declaring data.  The syntax is:
% 
%          typedef declaration;
% 
%    The new type name is actually the variable name in the declaration
%    part of the typedef.  For example, the following defines a new type
%    called "eggbox" using an existing type called "egg":
% 
%          typedef egg eggbox[DOZEN];
% 
%    Variables declared using the new type name have the same type as the
%    new type name would have in the typedef, if it were considered a
%    variable.  For example, the following two declarations are equivalent
%    in declaring the variable "fresheggs":
% 
%          eggbox  fresheggs; egg     fresheggs[DOZEN];
% 
%    When a typedef involves a struct, enum, or union definition, there is
%    another (preferred) syntax that may be used to define the same type.
%    In general, a typedef of the following form:
% 
%          typedef <<struct, union, or enum definition>> identifier;
% 
%    may be converted to the alternative form by removing the "typedef"
%    part and placing the identifier after the "struct", "union", or
%    "enum" keyword, instead of at the end.  For example, here are the two
%    ways to define the type "bool":
% 
%          typedef enum {    /* using typedef */
%             FALSE = 0,
%             TRUE = 1
%          } bool;
% 
%          enum bool {       /* preferred alternative */
%             FALSE = 0,
%             TRUE = 1
%          };
% 
%    This syntax is preferred because one does not have to wait until the
%    end of a declaration to figure out the name of the new type.

% 4.19.  Optional-Data
\section{داده اختیاری}

%    Optional-data is one kind of union that occurs so frequently that we
%    give it a special syntax of its own for declaring it.  It is declared
%    as follows:
% 
%          type-name *identifier;
% 
%    This is equivalent to the following union:
% 
%          union switch (bool opted) {
%          case TRUE:
%             type-name element;
%          case FALSE:
%             void;
%          } identifier;

%    It is also equivalent to the following variable-length array
%    declaration, since the boolean "opted" can be interpreted as the
%    length of the array:
% 
%          type-name identifier<1>;
% 
%    Optional-data is not so interesting in itself, but it is very useful
%    for describing recursive data-structures such as linked-lists and
%    trees.  For example, the following defines a type "stringlist" that
%    encodes lists of zero or more arbitrary length strings:
% 
%         struct stringentry {
%            string item<>;
%            stringentry *next;
%         };
% 
%         typedef stringentry *stringlist;
% 
%    It could have been equivalently declared as the following union:
% 
%          union stringlist switch (bool opted) {
%          case TRUE:
%             struct {
%                string item<>;
%                stringlist next;
%             } element;
%          case FALSE:
%             void;
%          };
% 
%    or as a variable-length array:
% 
%         struct stringentry {
%            string item<>;
%            stringentry next<1>;
%         };
% 
%         typedef stringentry stringlist<1>;
% 
%    Both of these declarations obscure the intention of the stringlist
%    type, so the optional-data declaration is preferred over both of
%    them.  The optional-data type also has a close correlation to how
%    recursive data structures are represented in high-level languages
%    such as Pascal or C by use of pointers.  In fact, the syntax is the
%    same as that of the C language for pointers.
% 