\documentclass{../boostan/Boostan-UserManual}

\title{ند}
\subtitle{نمایش داده‌های خارجی}
\type{}
\author{مصطفی برمشوری، محمدهادی منصوری، قاسم خانزاده}
\date{}
\version{1.0}
\project{سبز-۱}
\SetWatermarkText{ند}


\begin{document}

\newglossaryentry{agent}{%
	name={agent},%
	symbol={\lr{agent}},%
	plural={نهاد},%
	description={%
		عبارت است از هر موجودیتی که در سیستم یک شناسه داشته و سیستم به صورت دوره‌ای آن
		را ردیابی می‌کند. به طور معمول نهاد معادل با یک خودرو در نظر گرفته می‌شود.
	},
}

\newglossaryentry{row data}{%
	name={row data},%
	symbol={\lr{RD}},%
	plural={سطر داده},%
	description={%
		به مجموعه خصوصیت‌هایی که در یک تاریخ مشخص در مورد یک نهاد گردآوری شده است گفته
		می‌شود. این داده معمولا شامل تاریخ، مکان، سرعت و دما می‌شود.
	},
}

\newglossaryentry{big-endian}{%
	name={big-endian},%
	symbol={\lr{big-endian}},%
	plural={بیت انتهایی پر ارزش‌ترین},%
	description={},
}

\newglossaryentry{discriminated union}{%
	name={discriminated union},%
	symbol={\lr{big-endian}},%
	plural={گردایه گزینشی},%
	description={},
}



\maketitle

\begin{abstract}
این مستند، استاندارد نمایش داده (ند) را توصیف می‌کند. این استاندارد در حال حاضر
به عنوان مرجع طراحی فیزیکی در سیستم‌های رایانه‌ای به کار گرفته می‌شود. این مستند
به عنوان سند سبر-۱ معرفی شده است.
\end{abstract}
\clearpage

\pagenumbering{alph}
\tableofcontents
\listoffigures 
\listoftables

\pagenumbering{arabic}

\chapter{دیباچه}
% 1.  Introduction

%    XDR is a standard for the description and encoding of data.  It is
%    useful for transferring data between different computer
%    architectures, and it has been used to communicate data between such
%    diverse machines as the SUN WORKSTATION*, VAX*, IBM-PC*, and Cray*.
%    XDR fits into the ISO presentation layer and is roughly analogous in
%    purpose to X.409, ISO Abstract Syntax Notation.  The major difference
%    between these two is that XDR uses implicit typing, while X.409 uses
%    explicit typing.

ند یک استاندارد برای توصیف روش‌های کدگذاری است که در سیستم‌های پردازشی مورد
استفاده قرار می‌گیرد.
ند برای انتقال داده‌ها در شبکه‌های رایانه‌ای و یا سخت‌افزارها، که در آنها معماری
نهادهای موجود در ارتباط با یکدیگر متفاوت است بسیار کاربرد دارد.
برای نمونه حالتی تصور کنید که در آن رایانه‌ای شخصی با استفاده از گذرگاه‌های
\lr{PCI} با یک ریزپردازنده در ارتباط است. 
بدیهی است که معماری این درو پردازنده با یکدیگر متفاوت است و استفاده از ساختارهای
داده عمومی نمی‌تواند در ارتباط بین آنها کاربرد داشته باشد.
ند استانداردی است که کدگذاری این نوع ارتباط را تعریف می‌کند.
استاندارد ند مشابه با استانداردهای \lr{XDR}\cite{srinivasan1995xdr} و \lr{X.409}\cite{x409}
است.
\lr{X.409} که در لایه نمایش پشته پرتوکلی \lr{ISO} کاربرد دارد، انواع داده را به
صورت صریح بیان می‌کند.
این درحالی است که استاندارد \lr{XDR} انواع داده را به صورت مجازی تعریف می‌کند و
در استانداردهای متفاوت مانند \lr{IEE1609} به کار گرفته شده است.
استاندارد ند نیز مانند استاندارد \lr{XDR} داده‌ها را به صورت مجازی نمایش می‌دهد
با این تفاوت که مدل نمایش فیزیکی داده در این استاندارد به ساختارهای فیزیکی و
پردازنده‌ها نزدیکتر است.

%    XDR uses a language to describe data formats.  The language can be
%    used only to describe data; it is not a programming language.  This
%    language allows one to describe intricate data formats in a concise
%    manner.  The alternative of using graphical representations (itself
%    an informal language) quickly becomes incomprehensible when faced
%    with complexity.  The XDR language itself is similar to the C
%    language [KERN], just as Courier [COUR] is similar to Mesa.
%    Protocols such as ONC RPC (Remote Procedure Call) and the NFS*
%    (Network File System) use XDR to describe the format of their data.

ند در حقیقت زبانی است که برای توصیف ساختارهای داده‌ای به کار گرفته می‌شود. این
زبان تنها برای توصیف داده به کار گرفته می‌شود و نمی‌توان آن را به عنوان یک زبان
برنامه سازی در نظر گرفت.
استفاده از زبان‌های گرافیکی برای نمایش داده‌ها، در قراردادها و ارتباط‌های فیزیکی
نارسایی‌های زیادی دارد.
برای نمونه در این نوع زبان‌ها نحوه قرار گرفتن داده‌ها به صورت یک جریان بیتی مشخص
نیست.
زبان ند بسیار شبیه به زبان برنامه سازی \lr{C} است با این تفاوت که ساختارهای
داده‌ای خاصی را توسعه داده و اصول یکتایی برای تبدیل این ساختارهای به رشته بیت را
مشخص کرده است.
این زبان در کاربردهایی مانند، قراردادهای شبکه، فراخوانی دور روال‌ها، انتقال داده
بین دستگاه‌های سخت افزاری و محاسبات توزیعی کابرد فراوان دارد.

\section{فرض‌ها}

در استاندارد ند دسته‌ای از اصول به عنوان فرض اولیه در نظر گرفته شده است. در این
بخش فرض‌های این استاندار تشریح شده است.

کوچکترین واحد داده بایت (۸ بیت) در نظر گرفته می‌شود که نمایش آن در تمام
سخت‌افزارها و سیستم‌های نرم‌افزاری یکسان است.
از این رو تمام سخت افزارها باید یک بایت داده را به گونه‌ای کدگذاری کنند که در
تمام سخت افزارها به یک شکل کد گذاری می‌شود و نتوان از آن معنی دیگری برداشت کرد.
برای نمونه در پردازنده‌ها یک بایت به صورت \glspl{big-endian} کدگذاری می‌شود، از
این رو تمام سخت افزارهای دیگری که از این استاندارد استفاده می‌کنند باید این نوع
کدگذاری را استفاده کنند.

\begin{note}
این فرض به این معنی است که تمام نهادهایی که از این استاندارد استفاده می‌کنند
باید به صورت یکتا کدگزاری یک بایت داده را انجام دهند.
\end{note}


% 2.  Changes from RFC 1832
\section{تاریخچه}

%    This document makes no technical changes to RFC 1832 and is published
%    for the purposes of noting IANA considerations, augmenting security
%    considerations, and distinguishing normative from informative
%    references.

استاندارد ند یک توسعه از استاندارد \lr{XDR} است، که در آن اصول نمایش داده به
شکلی تغییر کرده که با ساختارهای پردازنده‌ها و ریزپردازنده‌ها منطبق باشد.

\begin{table}
\begin{centering}
\begin{tabular}{|c|r|p{5cm}|}\hline
	نسخه &
	تاریخ &
	تغییر\\\hline
	1.0 &
	آذر 1392 &
	ارائه اولیه قرارداد نمایش داده (ند)\\\hline
\end{tabular}
\label{introduction/history}
\caption[تاریخچه تغییرات سند]{
	تاریخچه تغییرات سند.
}
\end{centering}
\end{table}




\chapter{مبانی}

% 3.  Basic Block Size
% 
%    The representation of all items requires a multiple of four bytes (or
%    32 bits) of data.  The bytes are numbered 0 through n-1.  The bytes
%    are read or written to some byte stream such that byte m always
%    precedes byte m+1.  If the n bytes needed to contain the data are not
%    a multiple of four, then the n bytes are followed by enough (0 to 3)
%    residual zero bytes, r, to make the total byte count a multiple of 4.

تمام داده‌ها با استفاده از یک جریان بایتی نمایش داده می‌شود که شامل ضریبی از ۴
بایت است.
تمام بایت‌ها با استفاده از یک اندیس شماره گزاری می‌شود که از شماره $0$ تا $n-1$
است.
این نحوه شماره گزاری بیانگر این مطلب است که همواره بایت اول کم ارزش‌ترین بایت
و بایت انتهای با ارزش ترین بایت است.
هر زمان تعداد بایت‌های مورد نیاز برای نمایش داده مضربی از عدد چهار باشد، تعدادی
بایت با مقدارد صفر به انتهای آن اضافه می‌شود.
 
%    We include the familiar graphic box notation for illustration and
%    comparison.  In most illustrations, each box (delimited by a plus
%    sign at the 4 corners and vertical bars and dashes) depicts a byte.
%    Ellipses (...) between boxes show zero or more additional bytes where
%    required.

برای اینکه درک صحیح از نمایش جریان داده به وجود آید و همچنین مقایسه و درک
ساختارهای داده‌ایجاد شده راحتر باشد، از نمایش گرافیکی برای جریان داده‌ها استفاده
می‌شود.
در این نمایش داده‌های هر جعبه بیانگر یک بایت داده است،  و نقطه چین بین جعبه‌ها
بیانگر این مطلب است که تعدادی محدود بایت در این فضا قرار گرفته اشت.
در شکل \ref{image/basics/stream} یک نمونه نمایش گرافیکی داده آورده شده است.

%         +--------+--------+...+--------+--------+...+--------+
%         | byte 0 | byte 1 |...|byte n-1|    0   |...|    0   |   BLOCK
%         +--------+--------+...+--------+--------+...+--------+
%         |<-----------n bytes---------->|<------r bytes------>|
%         |<-----------n+r (where (n+r) mod 4 = 0)>----------->|

\begin{figure}
\centering
\includegraphics[width=0.8\textwidth]{image/basics/stream}
\caption[ساختار یک جریان داده]{
	جریان داده برای نمایش ساختارهای داده‌ای شامل تعدادی بایت می‌شود که اولین بایت
	کم ارزش‌ترین بایت در نظر گرفته می‌شود.
}
\label{image/basics/stream}
\end{figure}

% 4.  Data Types
\chapter{نوع داده‌}

%    Each of the sections that follow describes a data type defined in the
%    XDR standard, shows how it is declared in the language, and includes
%    a graphic illustration of its encoding.

در این بخش انواع داده‌های از پیش تعریف شده در استاندارد ند تشریح شده است.
هر ساختار داده با استفاده از شکل گرافیکی تشریح شده است که بیانگر نحوه ذخیره سازی
آن در جریان داده است.
 
%    For each data type in the language we show a general paradigm
%    declaration.  Note that angle brackets (< and >) denote variable-
%    length sequences of data and that square brackets ([ and ]) denote
%    fixed-length sequences of data.  "n", "m", and "r" denote integers.
%    For the full language specification and more formal definitions of
%    terms such as "identifier" and "declaration", refer to Section 6,
%    "The XDR Language Specification".
%
%    For some data types, more specific examples are included.  A more
%    extensive example of a data description is in Section 7, "An Example
%    of an XDR Data Description".



% 4.16.  Void
\section{تهی}

%    An XDR void is a 0-byte quantity.  Voids are useful for describing
%    operations that take no data as input or no data as output.  They are
%    also useful in unions, where some arms may contain data and others do
%    not.  The declaration is simply as follows:

تهی معادل با داده‌ای به طول صفر است. داده تهی برای توصیف عمل‌هایی که هیچ داده
خروجی و یا ورودی ندارند پر کاربرد است.
در برخی از موارد که قسمت‌هایی از ساختارهای داده‌ای باید خالی باشد، داده تهی
کاربرد دارد.
تعریف یک داده تهی به صورت زیر است:

\begin{C++}
void;
\end{C++}

% 
%    Voids are illustrated as follows:
% 
%            ++
%            ||                                                     VOID
%            ++
%          --><-- 0 bytes

\section{بایت}

کوچکترین واحد داده در استاندارد ند بایت است. در استاندارد هیچ کدگذاری برای یک
بایت در نظر گرفته نمی‌شود.
در حالت کلی یک بایت به صورت زیر تعریف می‌شود:

\begin{C++}
byte identifier;
\end{C++}

\section{بایت بدون علامت}

کوچکترین واحد داده در استاندارد ند بایت است که معادل با یک عدد صحیح بدون علامت
در نظر گرفته می‌شود.
در حالت کلی یک بایت به صورت زیر تعریف می‌شود:

\begin{C++}
unsigned byte identifier;
\end{C++}




% 4.1.  Integer
\section{عدد صحیح}

%    An XDR signed integer is a 32-bit datum that encodes an integer in
%    the range [-2147483648,2147483647].  The integer is represented in
%    two's complement notation.  The most and least significant bytes are
%    0 and 3, respectively.  Integers are declared as follows:

در استاندارد ند، عدد صیحیح علامت‌دار با استفاده از یک داده ۳۲ بیتی کد گذاری
می‌شود که قادر است عددهای صحیح در بازه $[-2147483648,2147483647]$ را در خود جای
دهد.
عدد صحیح با استفاده از متمم دو نمایش داده می‌شود.
پر ارزش‌ترین بایت بایت شماره ۳ و کم ‌ارزش‌ترین بایت بایت شماره صفر است.
در شکل \ref{image/data-types/integer} ساختار کلی جریان داده برای کدگزاری یک عدد
صحیح نمایش داده شده است.

%          int identifier;
% 
%            (MSB)                   (LSB)
%          +-------+-------+-------+-------+
%          |byte 0 |byte 1 |byte 2 |byte 3 |                      INTEGER
%          +-------+-------+-------+-------+
%          <------------32 bits------------>

\begin{figure}
\centering
\includegraphics[width=0.8\textwidth]{image/data-types/integer}
\caption[کدگذاری یک داده صحیح]{
	ساختار داده‌ای یک داده صحیح شامل ۴ بایت است که برای نمایش یک عدد صحیح ۳۲ بیتی
	به کار گرفته می‌شود. در این ساختار بایت صفر کم ارزش‌ترین و بایت ۳ پر ارزش‌ترین
	بایت است.
}
\label{image/data-types/integer}
\end{figure}


یک عدد صیحی به صورت زیر تعریف می‌شود:

\begin{C++}
int identifier;
\end{C++}

% 4.2.  Unsigned Integer
\section{عدد صحیح بدون علامت}

%    An XDR unsigned integer is a 32-bit datum that encodes a non-negative
%    integer in the range [0,4294967295].  It is represented by an
%    unsigned binary number whose most and least significant bytes are 0
%    and 3, respectively.  An unsigned integer is declared as follows:
% 
%          unsigned int identifier;
% 
%            (MSB)                   (LSB)
%             +-------+-------+-------+-------+
%             |byte 0 |byte 1 |byte 2 |byte 3 |           UNSIGNED INTEGER
%             +-------+-------+-------+-------+
%             <------------32 bits------------>


\begin{figure}
\centering
\includegraphics[width=0.8\textwidth]{image/data-types/unsigned-integer}
\caption[ساختار یک داده صحیح بدون علامت]{
	ساختار داده‌ای یک داده صحیح بدون علامت شامل ۴ بایت است. در این ساختار بایت صفر
	کم ارزش‌ترین و بایت ۳ پر ارزش‌ترین بایت است.
}
\label{image/data-types/unsigned-integer}
\end{figure}







% 4.3.  Enumeration
\section{شماره}

%    Enumerations have the same representation as signed integers.
%    Enumerations are handy for describing subsets of the integers.
%    Enumerated data is declared as follows:
% 
%          enum { name-identifier = constant, ... } identifier;
% 
%    For example, the three colors red, yellow, and blue could be
%    described by an enumerated type:
% 
%          enum { RED = 2, YELLOW = 3, BLUE = 5 } colors;
% 
%    It is an error to encode as an enum any integer other than those that
%    have been given assignments in the enum declaration.


% 4.4.  Boolean
\section{دودویی}

%    Booleans are important enough and occur frequently enough to warrant
%    their own explicit type in the standard.  Booleans are declared as
%    follows:
% 
%          bool identifier;
% 
%    This is equivalent to:
% 
%          enum { FALSE = 0, TRUE = 1 } identifier;


% 4.5.  Hyper Integer and Unsigned Hyper Integer
\section{صحیح بزرگ}

%    The standard also defines 64-bit (8-byte) numbers called hyper
%    integers and unsigned hyper integers.  Their representations are the
%    obvious extensions of integer and unsigned integer defined above.
%    They are represented in two's complement notation.  The most and
%    least significant bytes are 0 and 7, respectively.  Their
%    declarations:
% 
%    hyper identifier; unsigned hyper identifier;
% 
% 
%         (MSB)                                                   (LSB)
%       +-------+-------+-------+-------+-------+-------+-------+-------+
%       |byte 0 |byte 1 |byte 2 |byte 3 |byte 4 |byte 5 |byte 6 |byte 7 |
%       +-------+-------+-------+-------+-------+-------+-------+-------+
%       <----------------------------64 bits---------------------------->
%                                                  HYPER INTEGER
%                                                  UNSIGNED HYPER INTEGER



% 4.5.  Hyper Integer and Unsigned Hyper Integer
\section{صحیح بزرگ بدون علامت}

%    The standard also defines 64-bit (8-byte) numbers called hyper
%    integers and unsigned hyper integers.  Their representations are the
%    obvious extensions of integer and unsigned integer defined above.
%    They are represented in two's complement notation.  The most and
%    least significant bytes are 0 and 7, respectively.  Their
%    declarations:
% 
%    hyper identifier; unsigned hyper identifier;
% 
% 
%         (MSB)                                                   (LSB)
%       +-------+-------+-------+-------+-------+-------+-------+-------+
%       |byte 0 |byte 1 |byte 2 |byte 3 |byte 4 |byte 5 |byte 6 |byte 7 |
%       +-------+-------+-------+-------+-------+-------+-------+-------+
%       <----------------------------64 bits---------------------------->
%                                                  HYPER INTEGER
%                                                  UNSIGNED HYPER INTEGER


% 4.6.  Floating-Point
\section{ممیز شناور}

%    The standard defines the floating-point data type "float" (32 bits or
%    4 bytes).  The encoding used is the IEEE standard for normalized
%    single-precision floating-point numbers [IEEE].  The following three
%    fields describe the single-precision floating-point number:
% 
%       S: The sign of the number.  Values 0 and 1 represent positive and
%          negative, respectively.  One bit.
% 
%       E: The exponent of the number, base 2.  8 bits are devoted to this
%          field.  The exponent is biased by 127.
% 
%       F: The fractional part of the number's mantissa, base 2.  23 bits
%          are devoted to this field.
% 
%    Therefore, the floating-point number is described by:
% 
%          (-1)**S * 2**(E-Bias) * 1.F
% 
%    It is declared as follows:
% 
%          float identifier;
% 
%          +-------+-------+-------+-------+
%          |byte 0 |byte 1 |byte 2 |byte 3 |              SINGLE-PRECISION
%          S|   E   |           F          |         FLOATING-POINT NUMBER
%          +-------+-------+-------+-------+
%          1|<- 8 ->|<-------23 bits------>|
%          <------------32 bits------------>
% 
%    Just as the most and least significant bytes of a number are 0 and 3,
%    the most and least significant bits of a single-precision floating-
%    point number are 0 and 31.  The beginning bit (and most significant
%    bit) offsets of S, E, and F are 0, 1, and 9, respectively.  Note that
%    these numbers refer to the mathematical positions of the bits, and
%    NOT to their actual physical locations (which vary from medium to
%    medium).
% 
%    The IEEE specifications should be consulted concerning the encoding
%    for signed zero, signed infinity (overflow), and denormalized numbers
%    (underflow) [IEEE].  According to IEEE specifications, the "NaN" (not
%    a number) is system dependent and should not be interpreted within
%    XDR as anything other than "NaN".


% 4.7.  Double-Precision Floating-Point
\section{ممیز شناور دوبل}

%    The standard defines the encoding for the double-precision floating-
%    point data type "double" (64 bits or 8 bytes).  The encoding used is
%    the IEEE standard for normalized double-precision floating-point
%    numbers [IEEE].  The standard encodes the following three fields,
%    which describe the double-precision floating-point number:
% 
%       S: The sign of the number.  Values 0 and 1 represent positive and
%             negative, respectively.  One bit.
% 
%       E: The exponent of the number, base 2.  11 bits are devoted to
%             this field.  The exponent is biased by 1023.
% 
%       F: The fractional part of the number's mantissa, base 2.  52 bits
%             are devoted to this field.
% 
%    Therefore, the floating-point number is described by:
% 
%          (-1)**S * 2**(E-Bias) * 1.F
% 
%    It is declared as follows:
% 
%          double identifier;
% 
%          +------+------+------+------+------+------+------+------+
%          |byte 0|byte 1|byte 2|byte 3|byte 4|byte 5|byte 6|byte 7|
%          S|    E   |                    F                        |
%          +------+------+------+------+------+------+------+------+
%          1|<--11-->|<-----------------52 bits------------------->|
%          <-----------------------64 bits------------------------->
%                                         DOUBLE-PRECISION FLOATING-POINT
% 
%    Just as the most and least significant bytes of a number are 0 and 3,
%    the most and least significant bits of a double-precision floating-
%    point number are 0 and 63.  The beginning bit (and most significant
%    bit) offsets of S, E, and F are 0, 1, and 12, respectively.  Note
%    that these numbers refer to the mathematical positions of the bits,
%    and NOT to their actual physical locations (which vary from medium to
%    medium).

%    The IEEE specifications should be consulted concerning the encoding
%    for signed zero, signed infinity (overflow), and denormalized numbers
%    (underflow) [IEEE].  According to IEEE specifications, the "NaN" (not
%    a number) is system dependent and should not be interpreted within
%    XDR as anything other than "NaN".

% 4.8.  Quadruple-Precision Floating-Point
\section{ممیز شناور چهارلا}
% 
%    The standard defines the encoding for the quadruple-precision
%    floating-point data type "quadruple" (128 bits or 16 bytes).  The
%    encoding used is designed to be a simple analog of the encoding used
%    for single- and double-precision floating-point numbers using one
%    form of IEEE double extended precision.  The standard encodes the
%    following three fields, which describe the quadruple-precision
%    floating-point number:
% 
%       S: The sign of the number.  Values 0 and 1 represent positive and
%          negative, respectively.  One bit.
% 
%       E: The exponent of the number, base 2.  15 bits are devoted to
%          this field.  The exponent is biased by 16383.
% 
%       F: The fractional part of the number's mantissa, base 2.  112 bits
%          are devoted to this field.
% 
%    Therefore, the floating-point number is described by:
% 
%          (-1)**S * 2**(E-Bias) * 1.F
% 
%    It is declared as follows:
% 
%          quadruple identifier;
% 
%          +------+------+------+------+------+------+-...--+------+
%          |byte 0|byte 1|byte 2|byte 3|byte 4|byte 5| ...  |byte15|
%          S|    E       |                  F                      |
%          +------+------+------+------+------+------+-...--+------+
%          1|<----15---->|<-------------112 bits------------------>|
%          <-----------------------128 bits------------------------>
%                                       QUADRUPLE-PRECISION FLOATING-POINT
% 
%    Just as the most and least significant bytes of a number are 0 and 3,
%    the most and least significant bits of a quadruple-precision
%    floating-point number are 0 and 127.  The beginning bit (and most
%    significant bit) offsets of S, E , and F are 0, 1, and 16,
%    respectively.  Note that these numbers refer to the mathematical
%    positions of the bits, and NOT to their actual physical locations
%    (which vary from medium to medium).

%    The encoding for signed zero, signed infinity (overflow), and
%    denormalized numbers are analogs of the corresponding encodings for
%    single and double-precision floating-point numbers [SPAR], [HPRE].
%    The "NaN" encoding as it applies to quadruple-precision floating-
%    point numbers is system dependent and should not be interpreted
%    within XDR as anything other than "NaN".



% 4.17.  Constant
\section{ثابت}

% 
%    The data declaration for a constant follows this form:

تعریف یک داده ثابت به صورت زیر است:

\begin{C++}
const name-identifier = n;
\end{C++}

%    "const" is used to define a symbolic name for a constant; it does not
%    declare any data.  The symbolic constant may be used anywhere a
%    regular constant may be used.  For example, the following defines a
%    symbolic constant DOZEN, equal to 12.
ثابت برای تعریف یک مقدار ثابت به صورت یک سمبل است و هیچ نوع داده جدیدی را ایجاد
نمی‌کند.
یک ثاب را می‌توان هرجایی که یک مقدار ثابت کاربرد دارد استفاده کرد.
برای نمونه عبارت زیر یک مقدار ثابت برابر با عدد ۱۲ را تعریف می‌کند:

\begin{C++}
const DOZEN = 12;
\end{C++}



% 4.20.  Areas for Future Enhancement
% 
%    The XDR standard lacks representations for bit fields and bitmaps,
%    since the standard is based on bytes.  Also missing are packed (or
%    binary-coded) decimals.
% 
%    The intent of the XDR standard was not to describe every kind of data
%    that people have ever sent or will ever want to send from machine to
%    machine.  Rather, it only describes the most commonly used data-types
%    of high-level languages such as Pascal or C so that applications
%    written in these languages will be able to communicate easily over
%    some medium.
% 
%    One could imagine extensions to XDR that would let it describe almost
%    any existing protocol, such as TCP.  The minimum necessary for this
%    is support for different block sizes and byte-orders.  The XDR
%    discussed here could then be considered the 4-byte big-endian member
%    of a larger XDR family.


\chapter{آرایه}

% Fixed-Length Array
\section{آرایه با طول ثابت}

%    Declarations for fixed-length arrays of homogeneous elements are in
%    the following form:
% 
%          type-name identifier[n];
% 
%    Fixed-length arrays of elements numbered 0 through n-1 are encoded by
%    individually encoding the elements of the array in their natural
%    order, 0 through n-1.  Each element's size is a multiple of four
%    bytes.  Though all elements are of the same type, the elements may
%    have different sizes.  For example, in a fixed-length array of
%    strings, all elements are of type "string", yet each element will
%    vary in its length.
% 
%          +---+---+---+---+---+---+---+---+...+---+---+---+---+
%          |   element 0   |   element 1   |...|  element n-1  |
%          +---+---+---+---+---+---+---+---+...+---+---+---+---+
%          |<--------------------n elements------------------->|
% 
%                                                FIXED-LENGTH ARRAY

% Variable-Length Array
\section{آرایه با طول متغیر}

%    Counted arrays provide the ability to encode variable-length arrays
%    of homogeneous elements.  The array is encoded as the element count n
%    (an unsigned integer) followed by the encoding of each of the array's
%    elements, starting with element 0 and progressing through element
%    n-1.  The declaration for variable-length arrays follows this form:
آرایه با طول متغیر امکان کدگزاری داده با طول متغیر را فراهم می‌کند.
طول یک آرایه با استفاده از عدد صحیح بدون علامت نمایش داده می شود.
طول آرایه همواره با یک عدد صحیح بدون علامت نمایش داده می‌شود از این رو تنها
انواع داده بدون علامت و صحیح می‌تواند برای تعیین طول آرایه به کار رود.

\begin{C++}
type-name identifier<length-declaration>;
\end{C++}

انواع داده زیر می‌تواند به عنوان متغیر طول آرایه به کار گرفته شود:

\begin{itemize}
  \item \lr{unsigned byte}
  \item \lr{unsigned short int}
  \item \lr{unsigned int}
  \item \lr{unsigned long int}
\end{itemize}

برای نمونه یک آرایه از بایت‌ها که حداکثر ۲۵۵ بایت را در خود جای می‌دهد به صورت
زیر تعریف می‌شود:

\begin{C++}
byte data<unsigned byte length>;
\end{C++}

در کدگزاری یک آرایه با طول متغیر ابتدا طول آرایه کدگزاری می‌شود و در جریان بایتی
نوشته می‌شود، پس از آن هر عنصر آرایه به ترتیب اندیس آن کدگزاری شده و در جریان
بایتی نوشته می شود.
شکل \ref{image/array/variable-length} روش کدگزاری آرایه با طول متغیر را نمایش
داده است.

%            0  1  2  3
%          +--+--+--+--+--+--+--+--+--+--+--+--+...+--+--+--+--+
%          |     n     | element 0 | element 1 |...|element n-1|
%          +--+--+--+--+--+--+--+--+--+--+--+--+...+--+--+--+--+
%          |<-4 bytes->|<--------------n elements------------->|

\begin{figure}
\centering
\includegraphics[width=0.8\textwidth]{image/array/variable-length}
\caption[کدگذاری یک آرایه با طول متغیر]{
	کدگزاری یک آرایه با طول متغیر. متغیر طول آرایه در ابتدای جریان بایتی کدگزاری
	می‌شود و بعد از آن عنصرها به ترتیب اندیس کدگزاری خواهند شد.
}
\label{image/array/variable-length}
\end{figure}

\begin{warning}
استفاده از مقادیر غیر مجاز (بیش از گنجایش متغیر طول) و یا متغیرهای صحیح علامت
دار به عنوان طول خطا به شمار می‌آید.
\end{warning}


% 4.11.  String
\section{رشته}

%    The standard defines a string of n (numbered 0 through n-1) ASCII
%    bytes to be the number n encoded as an unsigned integer (as described
%    above), and followed by the n bytes of the string.  Byte m of the
%    string always precedes byte m+1 of the string, and byte 0 of the
%    string always follows the string's length.  If n is not a multiple of
%    four, then the n bytes are followed by enough (0 to 3) residual zero
%    bytes, r, to make the total byte count a multiple of four.  Counted
%    byte strings are declared as follows:
% 
%          string object<m>;
%       or
%          string object<>;
% 
%    The constant m denotes an upper bound of the number of bytes that a
%    string may contain.  If m is not specified, as in the second
%    declaration, it is assumed to be (2**32) - 1, the maximum length.
%    The constant m would normally be found in a protocol specification.
%    For example, a filing protocol may state that a file name can be no
%    longer than 255 bytes, as follows:
% 
%          string filename<255>;
% 
%             0     1     2     3     4     5   ...
%          +-----+-----+-----+-----+-----+-----+...+-----+-----+...+-----+
%          |        length n       |byte0|byte1|...| n-1 |  0  |...|  0  |
%          +-----+-----+-----+-----+-----+-----+...+-----+-----+...+-----+
%          |<-------4 bytes------->|<------n bytes------>|<---r bytes--->|
%                                  |<----n+r (where (n+r) mod 4 = 0)---->|
%                                                                   STRING
% 
%    It is an error to encode a length greater than the maximum described
%    in the specification.



% 4.14.  Structure
\section{ساختار}

%    The components of the structure are encoded in the order of their
%    declaration in the structure.  Each component's size is a multiple of
%    four bytes, though the components may be different sizes.
یک گردایه با معنی از داده‌ها به عنوان یک ساختار داده‌ای در نظر گرفته می‌شود.
در حالت کلی یک ساختار داده به صورت زیر تعریف می‌شود:

\begin{C++}
 struct {
    component-declaration-A;
    component-declaration-B;
    ...
 } identifier;
\end{C++}

تمام اجزای تشکیل دهنده یک ساختار داده‌ای بر اساس ترتیب تعریف شدن انها در ساختار
داده‌ای، کدگذاری می‌شوند.
اندازه هر بخش از ساختار داده‌ای می‌تواند متفاوت باشد، برای نمونه یک بایت داده به
همراه یک عدد صیحیح بدون علامت می‌تواند در این ساختار به کار گرفته شود.
در شکل \ref{image/struct/structuer} نحوه کدگذاری یک ساختار داده نمایش داده شده
است.

% 
%          +-------------+-------------+...
%          | component A | component B |...                      STRUCTURE
%          +-------------+-------------+...

\begin{figure}
\centering
\includegraphics[width=0.8\textwidth]{image/struct/structuer}
\caption[کدگذاری یک ساختار داده]{
	کدگذاری یک ساختار داده. هر بخش از یک ساختار داده به صورت مستقل کدگذاری شده و در
	نهایت کدگذاری‌های ایجاد شده بر اساس ترتیب تعریف هر بخش در یک کدگذاری جدید ترکیب
	می‌شود.
}
\label{image/struct/structuer}
\end{figure}


 

% 5.  Discussion
\chapter{نتیجه گیری}

%    (1) Why use a language for describing data?  What's wrong with
%        diagrams?
% 
%    There are many advantages in using a data-description language such
%    as XDR versus using diagrams.  Languages are more formal than
%    diagrams and lead to less ambiguous descriptions of data.  Languages
%    are also easier to understand and allow one to think of other issues
%    instead of the low-level details of bit encoding.  Also, there is a
%    close analogy between the types of XDR and a high-level language such
%    as C or Pascal.  This makes the implementation of XDR encoding and
%    decoding modules an easier task.  Finally, the language specification
%    itself is an ASCII string that can be passed from machine to machine
%    to perform on-the-fly data interpretation.
% 
%    (2) Why is there only one byte-order for an XDR unit?
% 
%    Supporting two byte-orderings requires a higher-level protocol for
%    determining in which byte-order the data is encoded.  Since XDR is
%    not a protocol, this can't be done.  The advantage of this, though,
%    is that data in XDR format can be written to a magnetic tape, for
%    example, and any machine will be able to interpret it, since no
%    higher-level protocol is necessary for determining the byte-order.
% 
%    (3) Why is the XDR byte-order big-endian instead of little-endian?
%        Isn't this unfair to little-endian machines such as the VAX(r),
%        which has to convert from one form to the other?
% 
%    Yes, it is unfair, but having only one byte-order means you have to
%    be unfair to somebody.  Many architectures, such as the Motorola
%    68000* and IBM 370*, support the big-endian byte-order.
% 
%    (4) Why is the XDR unit four bytes wide?
% 
%    There is a tradeoff in choosing the XDR unit size.  Choosing a small
%    size, such as two, makes the encoded data small, but causes alignment
%    problems for machines that aren't aligned on these boundaries.  A
%    large size, such as eight, means the data will be aligned on
%    virtually every machine, but causes the encoded data to grow too big.
%    We chose four as a compromise.  Four is big enough to support most
%    architectures efficiently, except for rare machines such as the
%    eight-byte-aligned Cray*.  Four is also small enough to keep the
%    encoded data restricted to a reasonable size.
% 
%    (5) Why must variable-length data be padded with zeros?
% 
%    It is desirable that the same data encode into the same thing on all
%    machines, so that encoded data can be meaningfully compared or
%    checksummed.  Forcing the padded bytes to be zero ensures this.
% 
%    (6) Why is there no explicit data-typing?
% 
%    Data-typing has a relatively high cost for what small advantages it
%    may have.  One cost is the expansion of data due to the inserted type
%    fields.  Another is the added cost of interpreting these type fields
%    and acting accordingly.  And most protocols already know what type
%    they expect, so data-typing supplies only redundant information.
%    However, one can still get the benefits of data-typing using XDR.
%    One way is to encode two things: first, a string that is the XDR data
%    description of the encoded data, and then the encoded data itself.
%    Another way is to assign a value to all the types in XDR, and then
%    define a universal type that takes this value as its discriminant and
%    for each value, describes the corresponding data type.


% 6.  The XDR Language Specification
\chapter{فنی}

% 6.1.  Notational Conventions
% 
%    This specification uses an extended Back-Naur Form notation for
%    describing the XDR language.  Here is a brief description of the
%    notation:
% 
%    (1) The characters '|', '(', ')', '[', ']', '"', and '*' are special.
%    (2) Terminal symbols are strings of any characters surrounded by
%    double quotes.  (3) Non-terminal symbols are strings of non-special
%    characters.  (4) Alternative items are separated by a vertical bar
%    ("|").  (5) Optional items are enclosed in brackets.  (6) Items are
%    grouped together by enclosing them in parentheses.  (7) A '*'
%    following an item means 0 or more occurrences of that item.
% 
%    For example, consider the following pattern:
% 
%          "a " "very" (", " "very")* [" cold " "and "]  " rainy "
%          ("day" | "night")
% 
%    An infinite number of strings match this pattern.  A few of them are:
% 
%          "a very rainy day"
%          "a very, very rainy day"
%          "a very cold and  rainy day"
%          "a very, very, very cold and  rainy night"
% 
% 6.2.  Lexical Notes
% 
%    (1) Comments begin with '/*' and terminate with '*/'.  (2) White
%    space serves to separate items and is otherwise ignored.  (3) An
%    identifier is a letter followed by an optional sequence of letters,
%    digits, or underbar ('\_').  The case of identifiers is not ignored.
%    (4) A decimal constant expresses a number in base 10 and is a
%    sequence of one or more decimal digits, where the first digit is not
%    a zero, and is optionally preceded by a minus-sign ('-').  (5) A
%    hexadecimal constant expresses a number in base 16, and must be
%    preceded by '0x', followed by one or hexadecimal digits ('A', 'B',
%    'C', 'D', E', 'F', 'a', 'b', 'c', 'd', 'e', 'f', '0', '1', '2', '3',
%    '4', '5', '6', '7', '8', '9').  (6) An octal constant expresses a
%    number in base 8, always leads with digit 0, and is a sequence of one
%    or more octal digits ('0', '1', '2', '3', '4', '5', '6', '7').
% 
% 6.3.  Syntax Information
% 
%       declaration:
%            type-specifier identifier
%          | type-specifier identifier "[" value "]"
%          | type-specifier identifier "<" [ value ] ">"
%          | "opaque" identifier "[" value "]"
%          | "opaque" identifier "<" [ value ] ">"
%          | "string" identifier "<" [ value ] ">"
%          | type-specifier "*" identifier
%          | "void"
% 
%       value:
%            constant
%          | identifier
% 
%       constant:
%          decimal-constant | hexadecimal-constant | octal-constant
% 
%       type-specifier:
%            [ "unsigned" ] "int"
%          | [ "unsigned" ] "hyper"
%          | "float"
%          | "double"
%          | "quadruple"
%          | "bool"
%          | enum-type-spec
%          | struct-type-spec
%          | union-type-spec
%          | identifier
% 
%       enum-type-spec:
%          "enum" enum-body
% 
%       enum-body:
%          "{"
%             ( identifier "=" value )
%             ( "," identifier "=" value )*
%          "}"
% 
%       struct-type-spec:
%          "struct" struct-body
% 
%       struct-body:
%          "{"
%             ( declaration ";" )
%             ( declaration ";" )*
%          "}"
% 
%       union-type-spec:
%          "union" union-body
% 
%       union-body:
%          "switch" "(" declaration ")" "{"
%             case-spec
%             case-spec *
%             [ "default" ":" declaration ";" ]
%          "}"
% 
%       case-spec:
%         ( "case" value ":")
%         ( "case" value ":") *
%         declaration ";"
% 
%       constant-def:
%          "const" identifier "=" constant ";"
% 
%       type-def:
%            "typedef" declaration ";"
%          | "enum" identifier enum-body ";"
%          | "struct" identifier struct-body ";"
%          | "union" identifier union-body ";"
% 
%       definition:
%            type-def
%          | constant-def
% 
%       specification:
%            definition *
% 
% 6.4.  Syntax Notes
% 
%    (1) The following are keywords and cannot be used as identifiers:
%    "bool", "case", "const", "default", "double", "quadruple", "enum",
%    "float", "hyper", "int", "opaque", "string", "struct", "switch",
%    "typedef", "union", "unsigned", and "void".
% 
%    (2) Only unsigned constants may be used as size specifications for
%    arrays.  If an identifier is used, it must have been declared
%    previously as an unsigned constant in a "const" definition.
% 
%    (3) Constant and type identifiers within the scope of a specification
%    are in the same name space and must be declared uniquely within this
%    scope.
% 
%    (4) Similarly, variable names must be unique within the scope of
%    struct and union declarations.  Nested struct and union declarations
%    create new scopes.
% 
%    (5) The discriminant of a union must be of a type that evaluates to
%    an integer.  That is, "int", "unsigned int", "bool", an enumerated
%    type, or any typedefed type that evaluates to one of these is legal.
%    Also, the case values must be one of the legal values of the
%    discriminant.  Finally, a case value may not be specified more than
%    once within the scope of a union declaration.








% 7.  An Example of an XDR Data Description
\chapter{نمونه}

% 
%    Here is a short XDR data description of a thing called a "file",
%    which might be used to transfer files from one machine to another.
% 
%          const MAXUSERNAME = 32;     /* max length of a user name */
%          const MAXFILELEN = 65535;   /* max length of a file      */
%          const MAXNAMELEN = 255;     /* max length of a file name */
% 
%          /*
%           * Types of files:
%           */
%          enum filekind {
%             TEXT = 0,       /* ascii data */
%             DATA = 1,       /* raw data   */
%             EXEC = 2        /* executable */
%          };
% 
%          /*
%           * File information, per kind of file:
%           */
%          union filetype switch (filekind kind) {
%          case TEXT:
%             void;                           /* no extra information */
%          case DATA:
%             string creator<MAXNAMELEN>;     /* data creator         */
%          case EXEC:
%             string interpretor<MAXNAMELEN>; /* program interpretor  */
%          };
% 
%          /*
%           * A complete file:
%           */
%          struct file {
%             string filename<MAXNAMELEN>; /* name of file    */
%             filetype type;               /* info about file */
%             string owner<MAXUSERNAME>;   /* owner of file   */
%             opaque data<MAXFILELEN>;     /* file data       */
%          };
% 
%    Suppose now that there is a user named "john" who wants to store his
%    lisp program "sillyprog" that contains just the data "(quit)".  His
%    file would be encoded as follows:
% 
% 
%        OFFSET  HEX BYTES       ASCII    COMMENTS
%        ------  ---------       -----    --------
%         0      00 00 00 09     ....     -- length of filename = 9
%         4      73 69 6c 6c     sill     -- filename characters
%         8      79 70 72 6f     ypro     -- ... and more characters ...
%        12      67 00 00 00     g...     -- ... and 3 zero-bytes of fill
%        16      00 00 00 02     ....     -- filekind is EXEC = 2
%        20      00 00 00 04     ....     -- length of interpretor = 4
%        24      6c 69 73 70     lisp     -- interpretor characters
%        28      00 00 00 04     ....     -- length of owner = 4
%        32      6a 6f 68 6e     john     -- owner characters
%        36      00 00 00 06     ....     -- length of file data = 6
%        40      28 71 75 69     (qui     -- file data bytes ...
%        44      74 29 00 00     t)..     -- ... and 2 zero-bytes of fill


% 8.  Security Considerations
\chapter{امنیت و ملاحظات آن}
% 
%    XDR is a data description language, not a protocol, and hence it does
%    not inherently give rise to any particular security considerations.
%    Protocols that carry XDR-formatted data, such as NFSv4, are
%    responsible for providing any necessary security services to secure
%    the data they transport.
% 
%    Care must be take to properly encode and decode data to avoid
%    attacks.  Known and avoidable risks include:
% 
%    *    Buffer overflow attacks.  Where feasible, protocols should be
%         defined with explicit limits (via the "<" [ value ] ">" notation
%         instead of "<" ">") on elements with variable-length data types.
%         Regardless of the feasibility of an explicit limit on the
%         variable length of an element of a given protocol, decoders need
%         to ensure the incoming size does not exceed the length of any
%         provisioned receiver buffers.
% 
%    *    Nul octets embedded in an encoded value of type string.  If the
%         decoder's native string format uses nul-terminated strings, then
%         the apparent size of the decoded object will be less than the
%         amount of memory allocated for the string.  Some memory
%         deallocation interfaces take a size argument.  The caller of the
%         deallocation interface would likely determine the size of the
%         string by counting to the location of the nul octet and adding
%         one.  This discrepancy can cause memory leakage (because less
%         memory is actually returned to the free pool than allocated),
%         leading to system failure and a denial of service attack.
% 
%    *    Decoding of characters in strings that are legal ASCII
%         characters but nonetheless are illegal for the intended
%         application.  For example, some operating systems treat the '/'
%         character as a component separator in path names.  For a
%         protocol that encodes a string in the argument to a file
%         creation operation, the decoder needs to ensure that '/' is not
%         inside the component name.  Otherwise, a file with an illegal
%         '/' in its name will be created, making it difficult to remove,
%         and is therefore a denial of service attack.
% 
%    *    Denial of service caused by recursive decoder or encoder
%         subroutines.  A recursive decoder or encoder might process data
%         that has a structured type with a member of type optional data
%         that directly or indirectly refers to the structured type (i.e.,
%         a linked list).  For example,
% 
%               struct m {
%                 int x;
%                 struct m *next;
%               };
% 
%         An encoder or decoder subroutine might be written to recursively
%         call itself each time another element of type "struct m" is
%         found.  An attacker could construct a long linked list of
%         "struct m" elements in the request or response, which then
%         causes a stack overflow on the decoder or encoder.  Decoders and
%         encoders should be written non-recursively or impose a limit on
%         list length.

\pagestyle{plain}
\bibliographystyle{ieeetr-fa}
\bibliography{bibtex}
\printindex
\printglossaries

\end{document}

% 
% 
% 
% 
% 
% 9.  IANA Considerations
% 
%    It is possible, if not likely, that new data types will be added to
%    XDR in the future.  The process for adding new types is via a
%    standards track RFC and not registration of new types with IANA.
%    Standards track RFCs that update or replace this document should be
%    documented as such in the RFC Editor's database of RFCs.
% 
% 10.  Trademarks and Owners
% 
%    SUN WORKSTATION  Sun Microsystems, Inc.
%    VAX              Hewlett-Packard Company
%    IBM-PC           International Business Machines Corporation
%    Cray             Cray Inc.
%    NFS              Sun Microsystems, Inc.
%    Ethernet         Xerox Corporation.
%    Motorola 68000   Motorola, Inc.
%    IBM 370          International Business Machines Corporation
% 
% 
% 
% 
% 
% 
% 
% Eisler                      Standards Track                    [Page 23]
% 
% RFC 4506       XDR: External Data Representation Standard       May 2006
% 
% 
% 11.  ANSI/IEEE Standard 754-1985
% 
%    The definition of NaNs, signed zero and infinity, and denormalized
%    numbers from [IEEE] is reproduced here for convenience.  The
%    definitions for quadruple-precision floating point numbers are
%    analogs of those for single and double-precision floating point
%    numbers and are defined in [IEEE].
% 
%    In the following, 'S' stands for the sign bit, 'E' for the exponent,
%    and 'F' for the fractional part.  The symbol 'u' stands for an
%    undefined bit (0 or 1).
% 
%    For single-precision floating point numbers:
% 
%     Type                  S (1 bit)   E (8 bits)    F (23 bits)
%     ----                  ---------   ----------    -----------
%     signalling NaN        u           255 (max)     .0uuuuu---u
%                                                     (with at least
%                                                      one 1 bit)
%     quiet NaN             u           255 (max)     .1uuuuu---u
% 
%     negative infinity     1           255 (max)     .000000---0
% 
%     positive infinity     0           255 (max)     .000000---0
% 
%     negative zero         1           0             .000000---0
% 
%     positive zero         0           0             .000000---0
% 
%    For double-precision floating point numbers:
% 
%     Type                  S (1 bit)   E (11 bits)   F (52 bits)
%     ----                  ---------   -----------   -----------
%     signalling NaN        u           2047 (max)    .0uuuuu---u
%                                                     (with at least
%                                                      one 1 bit)
%     quiet NaN             u           2047 (max)    .1uuuuu---u
% 
%     negative infinity     1           2047 (max)    .000000---0
% 
%     positive infinity     0           2047 (max)    .000000---0
% 
%     negative zero         1           0             .000000---0
% 
%     positive zero         0           0             .000000---0
% 
% 
% 
% 
% 
% 
% Eisler                      Standards Track                    [Page 24]
% 
% RFC 4506       XDR: External Data Representation Standard       May 2006
% 
% 
%    For quadruple-precision floating point numbers:
% 
%     Type                  S (1 bit)   E (15 bits)   F (112 bits)
%     ----                  ---------   -----------   ------------
%     signalling NaN        u           32767 (max)   .0uuuuu---u
%                                                     (with at least
%                                                      one 1 bit)
%     quiet NaN             u           32767 (max)   .1uuuuu---u
% 
%     negative infinity     1           32767 (max)   .000000---0
% 
%     positive infinity     0           32767 (max)   .000000---0
% 
%     negative zero         1           0             .000000---0
% 
%     positive zero         0           0             .000000---0
% 
%    Subnormal numbers are represented as follows:
% 
%     Precision            Exponent       Value
%     ---------            --------       -----
%     Single               0              (-1)**S * 2**(-126) * 0.F
% 
%     Double               0              (-1)**S * 2**(-1022) * 0.F
% 
%     Quadruple            0              (-1)**S * 2**(-16382) * 0.F
% 
% 12.  Normative References
% 
%    [IEEE]  "IEEE Standard for Binary Floating-Point Arithmetic",
%            ANSI/IEEE Standard 754-1985, Institute of Electrical and
%            Electronics Engineers, August 1985.
% 
% 13.  Informative References
% 
%    [KERN]  Brian W. Kernighan & Dennis M. Ritchie, "The C Programming
%            Language", Bell Laboratories, Murray Hill, New Jersey, 1978.
% 
%    [COHE]  Danny Cohen, "On Holy Wars and a Plea for Peace", IEEE
%            Computer, October 1981.
% 
%    [COUR]  "Courier: The Remote Procedure Call Protocol", XEROX
%            Corporation, XSIS 038112, December 1981.
% 
%    [SPAR]  "The SPARC Architecture Manual: Version 8", Prentice Hall,
%            ISBN 0-13-825001-4.
% 
%    [HPRE]  "HP Precision Architecture Handbook", June 1987, 5954-9906.

